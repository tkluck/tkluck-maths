%%%%%%%%%%%%%%%%%%%%%%%%%%%%%%%%%%%%%%%%%
% Jacobs Landscape Poster
% LaTeX Template
% Version 1.1 (14/06/14)
%
% Created by:
% Computational Physics and Biophysics Group, Jacobs University
% https://teamwork.jacobs-university.de:8443/confluence/display/CoPandBiG/LaTeX+Poster
% 
% Further modified by:
% Nathaniel Johnston (nathaniel@njohnston.ca)
%
% This template has been downloaded from:
% http://www.LaTeXTemplates.com
%
% License:
% CC BY-NC-SA 3.0 (http://creativecommons.org/licenses/by-nc-sa/3.0/)
%
%%%%%%%%%%%%%%%%%%%%%%%%%%%%%%%%%%%%%%%%%

%----------------------------------------------------------------------------------------
%	PACKAGES AND OTHER DOCUMENT CONFIGURATIONS
%----------------------------------------------------------------------------------------

\documentclass[final]{beamer}

\usepackage[orientation=portrait,scale=1.15]{beamerposter} % Use the beamerposter package for laying out the poster

\usepackage{ragged2e}

\usetheme{confposter} % Use the confposter theme supplied with this template

\setbeamercolor{block title}{fg=ngreen,bg=white} % Colors of the block titles
\setbeamercolor{block body}{fg=black,bg=white} % Colors of the body of blocks
\setbeamercolor{block alerted title}{fg=white,bg=dblue!70} % Colors of the highlighted block titles
\setbeamercolor{block alerted body}{fg=black,bg=dblue!10} % Colors of the body of highlighted blocks
% Many more colors are available for use in beamerthemeconfposter.sty

%-----------------------------------------------------------
% Define the column widths and overall poster size
% To set effective sepwid, onecolwid and twocolwid values, first choose how many columns you want and how much separation you want between columns
% In this template, the separation width chosen is 0.024 of the paper width and a 4-column layout
% onecolwid should therefore be (1-(# of columns+1)*sepwid)/# of columns e.g. (1-(4+1)*0.024)/4 = 0.22
% Set twocolwid to be (2*onecolwid)+sepwid = 0.464
% Set threecolwid to be (3*onecolwid)+2*sepwid = 0.708

\newlength{\sepwid}
\newlength{\onecolwid}
\newlength{\smallercolwid}
\setlength{\sepwid}{0.024\paperwidth} % Separation width (white space) between columns
\setlength{\onecolwid}{0.464\paperwidth} % Width of one column
\setlength{\smallercolwid}{0.45\paperwidth} % Width of one column
\setlength{\topmargin}{-0.5in} % Reduce the top margin size
%-----------------------------------------------------------

\usepackage{graphicx}  % Required for including images

\usepackage{booktabs} % Top and bottom rules for tables

%----------------------------------------------------------------------------------------
%	TITLE SECTION 
%----------------------------------------------------------------------------------------

\title{Computational Orbifold Equivalence} % Poster title

\author{Timo Kluck (supervisors: Gunther Cornelissen, Ana Ros Camacho)} % Author(s)

\institute{Utrecht University} % Institution(s)

%----------------------------------------------------------------------------------------

\begin{document}

\addtobeamertemplate{block end}{}{\vspace*{2ex}} % White space under blocks
\addtobeamertemplate{block alerted end}{}{\vspace*{2ex}} % White space under highlighted (alert) blocks

\setlength{\belowcaptionskip}{2ex} % White space under figures
\setlength\belowdisplayshortskip{2ex} % White space under equations

\begin{frame}[t] % The whole poster is enclosed in one beamer frame

\begin{alertblock}{Objectives}

\begin{columns}

\begin{column}{\sepwid}\end{column} % Empty spacer column

\begin{column}{\smallercolwid} % The first column

\justify

\textbf{\large{}What is orbifold equivalence?}{\large \par}

Two quasi-homogenous, multivariate polynomials $W(\vec{x})$
and $V(\vec{u})$ with finite-dimensional Jacobian are called \textbf{\emph{orbifold equivalent}} if there
exists a matrix factorization $Q$ of $W(\vec{x})-V(\vec{u})$
satisfying the following conditions: $\mathrm{qdim}_{W,\vec{x}}Q\neq0$
and $\mathrm{qdim}_{V,\vec{u}}Q\neq0$ (see below for definition).
It is quite involved: even the factorizations proving reflexiveness
are non-obvious. Example: for $x^{3}+xy^{3}$ we need

\[
\left(\begin{array}{cccc}
 &  & y^{3}+x^{2}+xu+u^{2} & y^{2}u+yuv+uv^{2}\\
 &  & v-y & x-u\\
x-u & -y^{2}u-yuv-uv^{2}\\
y-v & y^{3}+x^{2}+xu+u^{2}
\end{array}\right)
\]

The only known invariant for this equivalence is \textbf{\emph{central
charge}}: two orbifold equivalent quasi-homogeneous polynomials have an
equal value of $\sum_{i}(1 - |x_i|)$.
Orbifold equivalence has physical application in Landau-Ginzburg models.

\bigskip

\textbf{\large{}What is quantum dimension?}{\large \par}

The \textbf{\emph{quantum dimension }} associated to a matrix factorization $Q(\vec{x}, \vec{u})$ of $W(\vec{x}) - V(\vec{u})$ is given by

\[
\mathrm{qdim}_{W,\vec{x}}Q(\vec{x},\vec{u})=\pm\mathrm{res}\left(\frac{\mathrm{str}(\partial_{x_{1}}Q\cdots\partial_{x_{n}}Q\partial_{u_{1}}Q\cdots\partial_{u_{m}}Q)\mathrm{d}\vec{x}}{\partial_{x_{1}}W,\cdots,\partial_{x_{n}}W}\right)
\]

\end{column} % The first column
\begin{column}{\sepwid}\end{column} % Empty spacer column
\begin{column}{\onecolwid} % The first column

where the expression

\[
\mathrm{res}\left(\frac{f(\vec{x})\mathrm{d}\vec{x}}{g_{1}(\vec{x}),\cdots,g_{n}(\vec{x})}\right)
\]

is defined for all polynomials $f$, and all ordered sets of polynomials $g_1,\cdots,g_n$ such that $\mathbf{k}[\vec{x}]/(g_1,\cdots,g_n)$ is finite-dimensional, through the
following properties:

it vanishes if $f \in (g_1,\cdots,g_n)$;

linearity;

the normalization equation

\[
\mathrm{res}\left(\frac{\mathrm{d}\vec{x}}{x_{1},\cdots,x_{n}}\right)=1
\]
and the transformation rule

\[
\mathrm{res}\left(\frac{f(\vec{x})\det M(\vec{x})\mathrm{d}\vec{x}}{M_{1}^{j}(\vec{x})g_{j}(\vec{x}),\cdots,M_{n}^{j}(\vec{x})g_{j}(\vec{x})}\right)=\mathrm{res}\left(\frac{f(\vec{x})\mathrm{d}\vec{x}}{g_{1}(\vec{x}),\cdots,g_{n}(\vec{x})}\right)
\]
for any matrix $M(\vec{x})$ with polynomial matrix coefficients $M_i^j(\vec{x})$.

\bigskip

\textbf{\large{}What are we trying to accomplish?}{\large \par}

\emph{Establish a {\bf decision procedure} for whether two given polynomials
are orbifold equivalent.}

\end{column}
\end{columns}

\end{alertblock}

\begin{columns}[t] % The whole poster consists of two major columns

\begin{column}{\sepwid}\end{column} % Empty spacer column

\begin{column}{\onecolwid} % The first column

\begin{block}{Fully computational approach}

When fixing a rank for $Q$ and maximal degrees for its entries, the existence
of a solution $Q^{2}=(W-V)\mathrm{id}$ can be reduced to an ideal membership problem
in the polynomial ring generated by the coefficients of $Q$'s entries.
The ideal is generated by elements of \textbf{\emph{low total degree}}
(namely 2), but \textbf{\emph{in many variables}}. We are applying
advanced computational tools such as Julia and FGb
to this problem.

This search algorithm has a discrete
and a geometric part: first, enumerate all possible ranks for $Q$
and the maximal gradings for its coefficients (discrete). Then for
each rank \& maximal gradings, solve an ideal membership problem (geometric).
The geometric part is a finite computation, but the discrete enumeration
is not. \textbf{\emph{Can the rank and the gradings be bounded theoretically?}}
\end{block}

\vspace{60mm}

\begin{block}{References}

\nocite{*} % Insert publications even if they are not cited in the poster
\small{\bibliographystyle{unsrt}
\bibliography{sample}\vspace{0.75in}}

\end{block}

\end{column} % End of the first column

\begin{column}{\sepwid}\end{column} % Empty spacer column

\begin{column}{\onecolwid} % Begin a column which is two columns wide (column 2)

\begin{block}{Deformation approach}

Many existing orbifold equivalences have been found using an \emph{ansatz}
in fewer variables. On the other
hand, formal deformations of matrix factorizations have been successfully
computed using computer algebra. So far, these deformations
were defined by leaving $W-V$ undeformed. \textbf{\emph{Is it possible
to obtain new orbifold equivalences by deforming $W-V$ in a similar
way?}}
\end{block}

\begin{block}{The two-variable case: Central charge pre-image}

In the two-variable case, there are various well-known families of
orbifold equivalences, e.g. $x^{2k}+y^{2}\sim x^{k}+xy^{2}$. However,
there is still many examples of polynomials with equal central charge
for which no orbifold equivalence is known, e.g. $x^{108}+y^{54}\sim x^{648}y+y^{38}x$.
Should we expect these to be orbifold equivalent? If so, can we compute
the equivalence? If not, does that give us a new invariant in addition
to the central charge?

So far, we have \textbf{\emph{succeeded in enumerating central charge
pre-images}}, which gives us a rich source of possibilities\textbf{\emph{
}}to investigate.
\end{block}
%
\begin{block}{Technical contributions and challenges}

{\bf Contributions:}
\begin{itemize}
\item We have succeeded in integrating FGb with Julia to combine expressiveness
with computational power;
\item For the search algorithm, we have an effective way to eliminate many
of the `discrete' possibilities by simple inspection.
\item For small degrees and up to two variables, we can reproduce known equivalences
through an exhaustive search.
\end{itemize}
{\bf Main challenges:}
\begin{itemize}
\item Can the ideal membership problem be made computationally tractable for new cases?
\end{itemize}
\begin{itemize}
\item If so, can we prove new conjectured orbifold equivalences?
\end{itemize}
\end{block}

%----------------------------------------------------------------------------------------
%	CONTACT INFORMATION
%----------------------------------------------------------------------------------------

\begin{alertblock}{Contact Information}

\begin{minipage}{25cm}
\begin{itemize}
\item \href{http://www.infty.nl/}{\color{black} http://www.infty.nl}
\item \href{mailto:tkluck@infty.nl}{\color{black} tkluck@infty.nl}
\end{itemize}
\end{minipage}
\begin{minipage}{2cm}
\includegraphics{UU-logoENGELS_CMYK©}
\end{minipage}

\end{alertblock}


%----------------------------------------------------------------------------------------

\end{column} % End of the second column

\begin{column}{\sepwid}\end{column} % Empty spacer column

\end{columns} % End of all the columns in the poster

\end{frame} % End of the enclosing frame

\end{document}
